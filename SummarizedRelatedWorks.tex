\documentclass{article}
\usepackage[a4paper, left=35mm, right=35mm]{geometry}

\usepackage{xcolor}
\usepackage{hyperref}
\hypersetup{
    colorlinks,
    linkcolor={red!50!black},
    citecolor={blue!50!black},
    urlcolor={blue!80!black}
}

\begin{document}

\subsubsection*{\href{https://www.sciencedirect.com/science/article/pii/S1474667017526646}{ESTRAC-III: An Expert System for Train Traffic Control in Disturbed Situations}}

Can't access. Referenced by Fuzzy Rule system.

\subsubsection*{\href{https://ieeexplore.ieee.org/document/208646/}{The fundamental issues of an expert system for urban traffic control}}

Irrelevant. Delves too deeply into details about car traffic, volume congestion and junctions.

\subsubsection*{\href{http://s-space.snu.ac.kr/bitstream/10371/90495/1/5.Knowledge-Based_Expert_System_in_Traffic_Signal_Control_Systems..pdf}{Knowledge-Based Expert System in Traffic Signal Control Systems}}

Not yet read

\subsubsection*{\href{https://www.slideshare.net/KartikShenoy1/expert-system-automated-traffic-light-control-based-on-road-congestion}{Expert System - Automated Traffic Light Control Based on Road Congestion}}
A powerpoint presentation..?

\subsubsection*{\href{https://www.dcce.ibilce.unesp.br/~norian/cursos/mds/estudo_de_caso/FuzzyTrafficControl.pdf}{Fuzzy Rule-based Expert System for Real-Time Train Traffic Control}}

Relies on human dispatcher decision making. Tries to implement rule-based decision making, incorporating ``Expert Knowledge'' heuristics from operators modeled using fuzzy sets, and finaly simulate the impact of proposed decisions to be made to be able to give the best recommendations possible. Seems to create a node mesh of variables, probabilities and conditions, much akin to a neural network, but tuned by railway operators ``Expert Knowledge'' instead of iteratively by a computer.

\subsubsection*{\href{https://en.wikipedia.org/wiki/Fuzzy_set}{Fuzzy Set}}

Sets where their members are graded from 0 to 1, where 0 is not member at all and 1 is full member!

\subsubsection*{\href{https://ac.els-cdn.com/S0191261506000737/1-s2.0-S0191261506000737-main.pdf?_tid=02dc28ac-ceb5-4ec5-a54e-4c6d32ce7c62&acdnat=1530008179_14f86ec30be353025bc8af4b00094329}{Single-track train timetabling with guaranteed
optimality: Branch-and-bound algorithms
with enhanced lower bounds}}

Way too hard to read!

\subsection*{\href{https://link.springer.com/content/pdf/10.1007\%2Fs10951-018-0558-0.pdf}{An efficient train scheduling algorithm on a single-track railway system}}

Evaluates algoritm based on ``free-running time'' minus actual time. Seems to expand upon an earlier algorithm, trying to determine whether the path ahead is clear or not. Checks paths ahead for trains and their states. Works using heuristics. Only considers lines of rails, not a network.

\end{document}
